\documentclass{llncs}
\usepackage{graphicx}
\usepackage{float}
\usepackage{amssymb,amsmath}

\begin{document}


\title{Probabilistic and frequency automata on omega-words
\thanks{%
The research was supported by  Project 2009/0216/1DP/1.1.1.2.0/09/IPIA/VIA/044
from the 
European Social Fund.
} 
}

\author{
Ieva Ruk\v s\= ane,
Rihards Kri\v slauks,
Taisia Mischenko-Slatenkova,\\
Ilze Dzelme-B\= erzi\c na,
R\= usi\c n\v s Freivalds,
Ieva N\= agele}
\institute{Institute of Mathematics and Computer Science,
 University of Latvia,\\ Rai\c na bulv\= aris 29, Riga, LV-1459, Latvia
}

\maketitle

\begin{abstract}  Frequency computation was introduced in 1960-ies in an attempt to show that many properties of probabilistic algorithms can be simulated by purely deterministic computational devices. The aim of this paper is to show that for automata working on infinite input tapes the situation is very much different. Technically our main result shows that there is an $\omega$-language recognized with a bounded error by a 2-way probabilistic finite automaton but not by any 1-way probabilistic finite automata. On the other hand, such a distinction is not possible for frequency finite automata.

\end{abstract} 

\section{Introduction}


Probabilistic (randomized) algorithms is one of central notions in Theory of Computation. However, since long ago computer scientists have attempted to develop notions and technical implementations of these notions that would be similar to but not equal to randomization.
 

The notion of frequency computation was introduced by G.~Rose \cite{R60} as an attempt to have an absolutely deterministic mechanism with properties similar to probabilistic algorithms. The definition was as follows. A function $f$: $w \to w$ is $(m,n)$-computable, where $1 \leq m \leq n$, iff there exists a recursive function $R$: $w^n \to w^n$ such that, for all $n$-tuples $(x_1, \cdots , x_n)$ of distinct natural numbers,
$$
card\{i: (R(x_1,\cdots , x_n))_i = f(x_i)\} \geq m.
$$

R. McNaughton cites in his survey \cite{M61} a problem (posed by J.~Myhill) whether $f$ has  to be recursive if $m$ is close to $n$. This problem was answered by B.A.~Trakhtenbrot \cite{T64} by showing that $f$ is recursive whenever $2m > n$. On the other hand, B.A.~Trakhtenbrot \cite{T64} proved that if $2m = n$ then nonrecursive functions can be $(m,n)$-computed. E.B.~Kinber extended the research  by considering frequency enumeration of sets \cite{K72}. The class of $(m, n)$-computable sets equals the class of recursive sets if and only if $2m > n$. The notion of frequency computation can be extended to other models of computation.
Frequency computation in polynomial time was discussed in full detail by M.Hinrichs and G.Wechsung \cite{HW97}. 


 For resource bounded computations, the behavior of frequency computability is completely different: for e.g., whenever $n' - m' > n - m$, it is known that under any reasonable resource bound there are sets $(m', n')$-computable, but not $(m, n)$-computable. However, scaling down to finite automata, the analogue of Trakhtenbrot's result holds again: We show here that the class of languages $(m, n)$-recognizable by deterministic finite automata equals the class of regular languages if and only if $2m > n$.   Conversely, for
  $2m > n$, the class of languages $(m, n)$-recognizable by deterministic finite automata \cite{ADHP05} is uncountable for a two-letter alphabet. When restricted to a one-letter alphabet, then every $(m,n)$-recognizable language is regular. This was also shown by Kinber.
  
Frequency computations became increasingly popular when relation between frequency computation and computation with a small number of queries was discovered \cite{K92,HKO92,BGK96,CKKK97}.   However, there still remains a fundamental problem: is there a class of automata $A$ such that probabilistic $A$-automata can perform an action such that no frequency automaton can, and is there a class of automata $B$ such that frequency $A$-automata can perform an action such that no probabilistic automaton can.







The study of finite state automata working on infinite words was initiated by B\"uchi \cite{Bu60}. B\"uchi discovered connection between formulas of the monadic second order logic of infinite sequences (S1S) and $\omega$-regular languages, the class of languages over infinite words accepted by finite state automata. Few years later, Muller proposed an alternative definition of finite state automata on infinite words \cite{Mu63}. McNaughton proved that with Muller's definition, deterministic finite state automata recognize all $\omega$-regular languages \cite{Mc66}. Later, Rabin extended the decidability result of B\"uchi for S1S to the monadic second order of the infinite binary tree (S2S) \cite{Ra69}. Rabin theorem can be used to settle a number of decision problems in logic. A theory of automata over infinite words has started from these studies.

In the paper, we study probabilistic and frequency finite state automata over infinite words. The probabilistic variants of the finite state automata over infinite words have been recently introduced \cite{BG05}.

\section{Definitions and notations}
%
\subsection{Notation}
%
This section is devoted to the notations used in the paper. $\Sigma$ denotes a finite set of symbols called alphabet. A word is a sequence of symbols from $\Sigma$. The set $\Sigma^{*}$ represents the set of all finite words over alphabet $\Sigma$, where $\Sigma^{\omega}$ is the set of all infinite words over alphabet $\Sigma$. For infinite word $\alpha$ we write $\alpha = \alpha(0) \alpha(1) \alpha(2)...$ with $\alpha(i) \in \Sigma$. A set of infinite words over a given alphabet is called an $\omega$-language.

The number of occurrences of the symbol $a$ in an infinite word ($\omega$-word) $\alpha$ is denoted by $|\alpha|_a$. For a given $\omega$-word $\alpha \in \Sigma^{\omega}$, let $Occ(\alpha)=\{a \in \Sigma | \exists i, \alpha(i) = a\}$ be the finite set of symbols occurring in $\alpha$ and $Inf(\alpha) = \{ a \in \Sigma | \exists^\omega j, \alpha(j) = a\}$  where $\exists^\omega$ denotes the quantifier "there exists infinite many".
%
\begin{definition}
\textbf{$\omega$-languages}. Let $U \subseteq \Sigma^{*}$ be a language of finite strings. The limit $U$, $lim(U) = \{\alpha \in \Sigma^{\omega} | \exists^{\omega} n \in N_0: \alpha[0..n] \in U\}$.
\end{definition}

An $\omega $-word belongs to $lim(U)$ iff it has infinitely many prefixes in U.
%
\subsection{Classical  automata}
%
In this section, first the basic concepts of $\omega$-automata are presented (for more details refer to \cite{SP02}, \cite{T97}).
%
\begin{definition}
An \textbf{$\omega$-automaton} is a quintuple $A = (Q, \Sigma, \delta, q_0, Acc)$, where
\begin{itemize}
\item $Q$ is a finite set of states,
\item $\Sigma$ is a finite alphabet,
\item $\delta : Q \times \Sigma \rightarrow 2^{Q}$ is the state transition function,
\item $q_0 \in Q$ is the initial state,
\item $Acc$ is the acceptance component. 
\end{itemize}
In a deterministic $\omega$-automaton, a transaction function $\delta : Q \times \Sigma \rightarrow Q$ is used.
\end{definition}

The acceptance condition $Acc$ defines which of the infinite runs are accepting; it can be given in different ways. Usually three types of acceptance conditions are considered: B\"uchi acceptance condition, Streett acceptance condition and Rabin acceptance condition which are formally defined in \cite{BG05,SP02}.

\bigskip

\section{Results}

R.Freivalds \cite{F81} proved that the language $\{0^n1^n\}$ can be recognized by 2-way probabilistic finite automata with an arbitrarily high probability
 $1 - \epsilon $. Our main result considers the language
$$
L = \{0^{m_1}1^{n_1}20^{m_2}1^{n_2}2\cdots \}
$$
where  each word in $L$ contains infinitely many symbols 2, and 
$$
\sum_{k \in R}{2^{-n_k}}
$$
diverges where $R = \{ k | m_k = n_k \}$.
  
 
 
 
\bigskip 
 
\begin{theorem}
(1) There exists a 2-way probabilistic finite automaton recognizing  the language $L$ with probability 1.\\
(2) There exists no 1-way probabilistic finite automaton recognizing  the language $L$ with a bounded error.
\end{theorem}


\bigskip

The proof of this Theorem is based on a classical result from theory of probabilities.

\bigskip


\noindent {\bf Borel-Cantelli lemma.}
If $E_1, E_2, \ldots $ is a sequence of independent random events with probabilities  
$p_1, p_2, \ldots $ then:\\
-if $\sum _i p_i$ diverges then with probability 1 infinitely many of the events $E_1, E_2, \ldots $ have a positive outcome,\\
-if $\sum _i p_i$ converges then with probability 1 only a finite number of the events $E_1, E_2, \ldots $ has a positive outcome.


\bigskip



{\bf Proof of Theorem 1.} (1)  Let $c(\epsilon )$ and $d(\epsilon )$ be large natural numbers such that
$$ 
2(\frac{1}{2})^{d(\epsilon )} < \epsilon \mbox{       ,        }  (\frac{2^{c(\epsilon )}}{1+2^{c(\epsilon )}})^{d(\epsilon )} > 1-\epsilon .
$$
Let the input substring $x$ be of a form $0^{m_k}1^{n_k}$. The automaton processes alternately the block of zeros and the block of ones.
One processing of a block is a series of options when $c(\epsilon )$ random symbols 0 or 1 are produced per every letter in the block.
We call the processing to be positive if all the results are 1, and negative otherwise. If the length of the block is $n$ then the probability of a 
positive processing of it equals $2^{-n.c(\epsilon )}$.

We interpret a processing of an ordered pair of blocks $0^m1^n$ as a competition. A competition where one processing is positive and the other one
is negative is interpreted as a win of the block processed positively.

The automaton holds competitions until the total number of wins for the given blocks of zeros and ones reaches $d(\epsilon )$. If at this moment  the
blocks $0^{m_k}$ and $1^{n_k}$ has at least one win each, our $\omega $-automaton enters a special state $q_E$. If all the wins belong to one of the blocks
$0^{m_k}$ or $1^{n_k}$, then our $\omega $-automaton enters a special state $q_F$. After that the $\omega $-automaton reads the symbol 2, enters a spacial state $q_S$, goes to competitions of the blocks $0^{m_{k+1}}$ and $1^{n_{k+1}}$ and never returns to blocks with smaller indices.

A set of states of this  $\omega $-automaton is declared to be accepting iff it contains the states $q_S$ and $q_E$ but does not contain the state $q_F$.

Let an $\omega $-word be in the $\omega $-language $L$. Then it contains infinitely many symbols 2. Hence the state $q_S$ is entered infinitely often. 
Additionally,
 where $R = \{ k | m_k = n_k \}$. Hence by Borel-Cantelli lemma, the state $q_E$ is entered infinitely often with probability 1.

$$
\sum_{k \in R}{2^{-n_k}}
$$
diverges
Let an $\omega $-word be not in the $\omega $-language $L$ because it contains only a finite number of symbols 2. Then the state $q_S$ is not entered infinitely often.

Let an $\omega $-word be not in the $\omega $-language $L$ because
$$
\sum_{k \in R}{2^{-n_k}}
$$
diverges. Then by Borel-Cantelli lemma, the state $q_E$ by Borel-Cantelli lemma, the state $q_E$ is entered infinitely often with probability 1 is entered only a finite number of times. 

\bigskip


(2) Assume on the contrary that there exists a 1-way probabilistic finite
automaton $A$ recognizing $L$ with a probability $1 - \delta $ where $\frac{1}{2} > \delta > 0$. Let $r$ be a natural number such that 
$1-(1+\frac{1}{r}) \delta >\frac{1}{2}$.
Consider an $\omega $-word
$x = 0^{m_1}1^{n_1}20^{m_2}1^{n_2}2\cdots 20^{m_s}1^{n_s}2\cdots \in L$. 
For arbitrary moment $t$ of the processing $x$ by $A$ denote by $p_1(t), p_2(t), \cdots , p_k(t)$ the distribution of probabilities
of the states $q_1, q_2, \cdots , q_k$ of the automaton $A$. 

Now consider a sequence of $\omega$-words differing from $x$ only in the $s$-th subword $0^{m_s}1^{n_s}$:
$$
x_1 = 0^{m_1}1^{n_1}20^{m_2}1^{n_2}2\cdots 20^{m_s+1}1^{n_s+1}2\cdots 
$$
$$
x_2 = 0^{m_1}1^{n_1}20^{m_2}1^{n_2}2\cdots 20^{m_s+2}1^{n_s+2}2\cdots 
$$
$$
\cdots 
$$
$$
x_u = 0^{m_1}1^{n_1}20^{m_2}1^{n_2}2\cdots 20^{m_s+u}1^{n_s+u}2\cdots 
$$
$$
\cdots \\
$$
Since $x \in L$, all these $\omega $-words are in $L$ and hence $A$ accepts them. Consider distribution of probabilities of the states
of $A$ at moments when $A$ crosses the gap between $0^{m_s+u}$ and $1^{n_s+u}$. Since all $p_i(t)$ are real numbers between 0 and 1, and there are infinitely many $\omega $-words $x_u$, there must exist an accumulation point $p_1(\infty ), p_2(\infty ), \cdots , p_k(\infty )$ such that for arbitrary
$\epsilon > 0$ there exists a $u$ such that the distance between $p_1(u), p_2(u), \cdots , p_k(u)$ and $p_1(\infty ), p_2(\infty ), \cdots , p_k(\infty )$ is less than $\epsilon $. Hence there are two distinct  $u_1$ and $u_2$ such that the distance between the distributions $p_1(u_1), p_2(_1), \cdots , p_k(u_1)$
and $p_1(u_2), p_2(_2), \cdots , p_k(u_2)$ is less that $0^{m_s+u}1^{n_s+u}$. Hence the probability to accept the word $x_{\mbox{combined}}$ containing a combined subword $0^{m_s+u_1}1^{n_s+u_2}$ differs from the probability to accept $x_{u_1}$ or $x_{u_2}$ no more that for $\frac{\delta }{r}$.

The same construction in the fragment $0^{m_s+u+1}1^{n_s+u+1}$ using smaller value  $\delta ' = \frac{\delta }{2}$ gives another combined $\omega $-word where both the $u$-th and $u+1$-th fragments are corrupted. This way after infinitely many substitutions we get an $\omega $-word which is accepted with a probability 
$1- \frac{3}{2}\delta $ but the word is not in $L$ because it contains infinitely many fragments where $m_u \neq n_u$.

\qed



\begin{lemma} 
\label{L1}
For arbitrary natural number $z$ there is natural number $w$ such that
if any 2-way deterministic finite automaton $A$ recognizes an $\omega $-language, has $z$ states and reads all symbols on the input tape, then there is a constant $c$ such that the automaton  crosses any point on the input tape
no more than $c$ times.
\end{lemma}

{\bf Proof.}  For arbitrary  2-way deterministic finite automaton recognizing an $\omega $-language and having $z$ states and an arbitrary point $t$ on the input tape, by $l(j)$ and $r(j)$  we define two reflection functions $\{1,2,\cdots ,z\} \to \{1,2,\cdots ,z\}$ describing the state $q_{l(j)}$ (or $q_{r(j)}$, resp.) in which the automaton returns to the point $t$ after being crossed in the state $q_j$ in the direction from the right to the left (or from the left to the right, resp.). It is easy to see that for a fixed automaton the function $l(j)$ depends only on the first $t$ symbols of the $\omega $-word and not on the tail of the word.

A {\em crossing sequence} of the automaton $A$ on an $\omega $-word $x$  at a point $t$ is the sequence of the states $q_{k_1}, q_{k_2}, \cdots , q_{k_s}$ at the moments when the automaton crosses the point $t$. If $q_{k_u} = q_{k_{u+2v}}$ the $q_{k_u} = q_{k_{u+2v}} =  q_{k_{u+4v}} = \cdots$ and the crossing sequence is infinitely long. This cannot happen if $A$ reads all symbols on the input tape. Hence $c \leq 2z$. \qed


\begin{lemma}
\label{L2} 
If the $\omega $-words $x$ and $y$ are such that they have points $t_1$ and $t_2$ where the crossing sequences of $A$ are exactly the same then the combined word $w$ obtained by taking the head of $x$ till the point $t_1$ and the tail of $y$ after the point $t_2$ then the crossing sequence of $A$ on $w$ at the point $t_1$ coinsides with the two crossing sequences of $A$ on $x$ and $y$ at the points
$t_1$ and $t_2$, respectively, and $A$ accepts $w$ iff $A$ accepts $y$.
\end{lemma}

{\bf Proof.} Immediate. \qed

\bigskip

\begin{theorem}
(1) There exists no 2-way deterministic finite automaton recognizing  the language $L$.\\
(2) There exists no 2-way frequency finite automaton recognizing  the language $L$ with frequency $(n,n)$.\\
(3) There exists no 2-way frequency finite automaton recognizing  the language $L$ with frequency $(m,n)$ where $m > \frac{n}{2}$.
\end{theorem}

{\bf Proof.} (1)  Assume from the contrary that there exists a 2-way deterministic finite
automaton $A$ recognizing $L$. 
Every automaton recognizing $L$ is to read all symbols on the input tape. By Lemma \ref{L1}, the length of every crossing sequence of $A$ does not exceed $c$. 
Consider an $\omega $-word
$x = 0^{m_1}1^{n_1}20^{m_2}1^{n_2}2\cdots 20^{m_s}1^{n_s}2\cdots \in L$. 
For arbitrary point $t$  denote by $p_1(t), p_2(t), \cdots , p_k(t)$ the crossing sequence of  $A$ at $t$. 

Now consider a sequence of $\omega$-words differing from $x$ only in the $s$-th subword $0^{m_s}1^{n_s}$:
$$
x_1 = 0^{m_1}1^{n_1}20^{m_2}1^{n_2}2\cdots 20^{m_s+1}1^{n_s+1}2\cdots 
$$
$$
x_2 = 0^{m_1}1^{n_1}20^{m_2}1^{n_2}2\cdots 20^{m_s+2}1^{n_s+2}2\cdots 
$$
$$
\cdots 
$$
$$
x_u = 0^{m_1}1^{n_1}20^{m_2}1^{n_2}2\cdots 20^{m_s+u}1^{n_s+u}2\cdots 
$$
$$
\cdots \\
$$
Since $x \in L$, all these $\omega $-words are in $L$ and hence $A$ accepts them. Consider the crossing sequences
of $A$ at points between $0^{m_s+u}$ and $1^{n_s+u}$. There are only a finite number of possible crossing sequences of length not exceeding $c$.
 Hence there are two distinct  $u_1$ and $u_2$ such that the crossing sequence is the same. Consider the word $x_{\mbox{combined}}$ containing a combined subword $0^{m_s+u_1}1^{n_s+u_2}$ but otherwise the same as the considered words. By Lemma \ref{L2} is also accepted by $A$. Repeating this construction for infinitely many values of $u$ we get an $\omega $-word which is accepted by $A$ but which is not in $L$. Contradiction.
 
\bigskip

(2) This part of our Theorem is not implied immediately from (1) because 
even a $(2,2)$-automaton can use the other tape as a counter. For instance, the $\omega $-language
$$
M = \{0^{m_1}1^{n_1}20^{m_2}1^{n_2}2\cdots \mid (\exists b)(\forall s) (m_s = n_s = b)\}
$$
can be recognized by a deterministic finite $(2,2)$-automaton but not by a deterministic finite $(1,1)$-automaton.
However, we prove below that $L$ cannot be recognized by any $(n,n)$-automaton. Indeed, 
assume from the contrary that  there exists a 2-way deterministic finite
$(n,n)$-automaton $A$ recognizing $L$. 
 
Consider a natural number $b$ , an $\omega $-word $x = 0^b1^b20^b1^b2\cdots 20^b1^b2\cdots $ and an $n$-tuple of $\omega $-words
$$
x_{01} = 0120^b1^b20^b1^b2\cdots 20^b1^b2\cdots \\
$$
$$
x_{02} = 0^21^220^b1^b20^b1^b2\cdots 20^b1^b2\cdots \\
$$
$$
\cdots
$$
$$
x_{0n} = 0^n1^n20^b1^b20^b1^b2\cdots 20^b1^b2\cdots \\
$$
These words are in $L$. Hence $A$ accepts all of them. It means that there is there is an accepting set of states $Q_1 = \{q_{k_1}, q_{k_2}, \cdots , q_{k_d}\}$
and a moment $t$ such that when $A$ processes this $n$-tuple of $\omega $-words after the moment $t_1$ only states from $Q_1$ are entered and each state from 
$Q_1$ is entered infinitely many times. Now we consider a natural number $u$ such that it is larger than the lengths of all initial fragments of $x_{01}, \cdots , x_{0n}$ observed by $A$ till the moment $t_1$. Denote by $y_{01}, y_{02}, \cdots , y_{0n}$ the initial fragments of $x_{01}, x_{02}, \cdots , x_{0n}$, respectively,
of the length $u$. Consider an infinite sequence of $n$-tuples of $\omega $-words which differ only in the first component but have the same $x_{02}, x_{03}, \cdots , x_{0n}$. The first component are, respectively,
$$
y_{01}0120^b1^b20^b1^b2\cdots 20^b1^b2\cdots \\
$$
$$
y_{01}0^21^220^b1^b20^b1^b2\cdots 20^b1^b2\cdots \\
$$
$$
\cdots
$$
$$
y_{01}0^s1^s20^b1^b20^b1^b2\cdots 20^b1^b2\cdots \\
$$
$$
\cdots
$$
Since the sequence is infinite but, by Lemma \ref{L2}, the possibilities for the crossing sequences are limited, there two distinct $n$-tuples in this sequence 
such that $A$ cannot distinguish between them and, moreover, enters only states from $Q_1$ are entered and each state from 
$Q_1$ is entered infinitely many times. Let the two $n$-tuples have indices $i$ and $j$ in our sequence. Consider a combined first component
$y_{01}0^i1^j20^b1^b20^b1^b2\cdots 20^b1^b2\cdots $. After the moment $t_1$ $A$ enters only the states from $Q_1$ on this combined $n$-tuple.

Now let $t_2$ be a moment after $t_1$ such that each state from 
$Q_1$ is entered at least once between moments $t_1$ and $t_2$. Again we consider an infinite sequence of $n$-tuples of $\omega $-words which differ only in one  component. However, now this component is the second component. 

This way, we cyclically change the components and in the limit we get an $n$-tuple of $\omega $-words which is accepted but each component of which has infinitely many subwords $0^m1^n$ where $m\neq n$. Contradiction.





Now consider a sequence of $\omega$-words differing from $x$ only in the $s$-th subword $0^{m_s}1^{n_s}$:
$$
x_1 = 0^{m_1}1^{n_1}20^{m_2}1^{n_2}2\cdots 20^{m_s+1}1^{n_s+1}2\cdots 
$$
$$
x_2 = 0^{m_1}1^{n_1}20^{m_2}1^{n_2}2\cdots 20^{m_s+2}1^{n_s+2}2\cdots 
$$
$$
\cdots 
$$
$$
x_u = 0^{m_1}1^{n_1}20^{m_2}1^{n_2}2\cdots 20^{m_s+u}1^{n_s+u}2\cdots 
$$
$$
\cdots \\
$$
Since $x \in L$, all these $\omega $-words are in $L$ and hence $A$ accepts them. 













\bigskip

(3) We use the easily observable property of $L$ showing that if an $\omega $-word $x$ is in $L$ then changing any finite initial fragment of $x$ we again obtain an $\omega $-word $y \in L$. Assume from the contrary that $L$ is recognized by  $(m,n)$-automaton $A$ such that $m > \frac{n}{2}$. We wish to construct an 
$(n,n)$-automaton for $L$. Given any $x$, we construct $n$ {\em distinct} $\omega $-words $x_1, x_2, \cdots , x_n$ by adding initial fragments
$012, 00112, 0001112, \cdots , 000\cdots 111\cdots 2$ to $x$, applying $A$ to them and producing the result which coincides with the majority of the results of
$A$ on $x_1, x_2, \cdots , x_n$. This is a contradiction since we already proved in (2) impossibility of such an automaton.\qed


\begin{thebibliography}{99}

\bibitem{ADHP05} 
Holger Austinat, Volker Diekert, Ulrich Hertrampf, Holger Petersen.
Regular frequency computations.
{\em Theoretical Computer Science,} vol. 330 No. 1, pp. 15--20, 2005.

\bibitem{BG05}
Christel Baier, Marcus Gr\" osser.
Recognizing $\omega$-regular languages with probabilistic automata.
Proc. of the 20th IEEE Symposium on Logic in Computer Science, pp. 137--146, 2005.

\bibitem{BGK96}
Richard Beigel, William I. Gasarch, Efim B. Kinber.
Frequency computation and bounded queries.
{\em Theoretical Computer Science,} vol. 163, No. 1/2, pp. 177--192, 1996. 

%
\bibitem{Bu60}
J.R.B\"uchi:
On a decision method in restricted second order arithmetic. 
Z. Math. Logik Grundlag. Math, {\bf 6} (1960) 66--92
%

\bibitem{CKKK97} 
John Case, Susanne Kaufmann, Efim B. Kinber,  Martin Kummer.
Learning recursive functions from approximations.
{\em Journal of Computer and System Sciences,} vol. 55, No. 1, pp. 183--196, 1997. 

\bibitem{F81}
R\= usi\c n\v s Freivalds.
Probabilistic Two-Way Machines.
{\em Lecture Notes in Computer Science,} vol 118, pp. 33--45, 1981. 

\iffalse

\bibitem{F82}%TODO: uz šo vajag atsauci
R.Freivalds.
On the growth of the number of states in result of the determinization of probabilistic finite automata.
\textit{Avtomatika i Vichislitel'naya Tekhnika}, No. 3, pp. 39--42, 1982 (Russian)


\bibitem{FK94}%TODO: uz šo vajag atsauci
R\= usi\c n\v s Freivalds, Marek Karpinski. 
Lower Space Bounds for Randomized Computation. 
{\em Lecture Notes in Computer Science,} 820:580-592, 1994. 



\bibitem{F91}%TODO: uz šo vajag atsauci
R\= usi\c n\v s Freivalds.
Complexity of probabilistic versus deterministic automata. 
{\em Lecture Notes in Computer Science,} vol. 502, pp. 565--613, 1991.


\bibitem{F91q}%TODO: uz šo vajag atsauci
R\= usi\c n\v s Freivalds.
Inductive Inference of Recursive Functions: Qualitative Theory.
{\em Lecture Notes in Computer Science,} vol. 502, pp. 77--110, 1991.

\bibitem{F91c}%TODO: uz šo vajag atsauci
R\= usi\c n\v s Freivalds, J\= anis B\= arzdi\c n\v s, K\= arlis Podnieks. 
Inductive Inference of Recursive Functions: Complexity Bounds.
{\em Lecture Notes in Computer Science,} vol. 502, pp. 111--155, 1991.


\bibitem{F98}%TODO: uz šo vajag atsauci
R\= usi\c n\v s Freivalds.
Models of computation, Riemann Hypothesis and classical mathematics.
{\em Lecture Notes in Computer Science,} vol.1521, pp. 89--106, 1998.

\bibitem{F08}%TODO: uz šo vajag atsauci
R\= usi\c n\v s Freivalds.
Non-constructive methods for finite probabilistic automata.
{\em International Journal of Foundations of Computer Science,} vol. 19, No. 3, pp.565--580, 2008.

\bibitem{F10}%TODO: uz šo vajag atsauci
R\= usi\c n\v s Freivalds.
Amount of nonconstructivity in finite automata. 
{\em Theoretical Computer Science,} vol. 411, No. 38-39, pp.3436--3443, 2010.

\fi

\bibitem{SP02}
E. Gr\"adel, W. Thomas, T. Wilke, editors: 
Automata, Logics, and Infinite Games: A Guide to Current Research.
Lecture Notes in Computer Science 2500. Springer. (2002)
%


\bibitem{HKO92}
Valentina Harizanova, Martin Kummer, Jim Owings.
Frequency computations and the cardinality theorem.
{\em The Journal of Symbolic Logic,} vol. 57, No. 2, pp. 682--687, 1992.

\bibitem{HW97}
Maren Hinrichs and Gerd Wechsung.
Time bounded frequency computations.
{\em Information and Computation,} vol. 139, pp. 234-257, 1997.



\bibitem{K72}
Efim B. Kinber.
Frequency calculations of general recursive predicates and frequency
enumeration of sets.
{\em Soviet Mathematics Doklady,} vol. 13, pp. 873--876, 1972.


\bibitem{K92}
Martin Kummer.
A proof of Beigel's Cardinality Conjecture. 
{\em The Journal of Symbolic Logic,} vol. 57, No. 2, pp. 677--681, 1992.

\bibitem{Mu63}
D.E. Muller. 
Infinite sequences an finite machines.
{\em Proc. 4th IEEE Symp. on Switching Circuit Theory and Logical Design,} pp. 3--16, 1963.

\bibitem{M61}
Robert McNaughton.
The Theory of Automata, a Survey.
{\em Advances in Computers,} vol. 2, pp. 379--421, 1961.

\bibitem{Mc66}
R. McNaughton: 
Testing and generating infinite sequences by a finite automaton.
Inform. Control, {\bf 9} (1966) 521--530
%

\bibitem{Ra69}
M.O. Rabin: 
Decidability of second order theories and automata on infinite trees.
Trans. AMS, {\bf 141} (1969) 1--37
%

\bibitem{R60}
Gene F. Rose.
An extended notion of computability.
{\em Abstracts of International Congress for Logic, Methodology and Philosophy of Science,} p.14, 1960.

\bibitem{T97}
W. Thomas: 
Languages, Automata, and Logic. 
Handbook of Formal Languages 3, (1997) 389--455.

\bibitem{T64}
Boris A. Trakhtenbrot.
On the frequency computation of functions.
{\em Algebra i Logika,} vol. 2, pp.25--32, 1964 (Russian)


\end{thebibliography}


\end{document}

